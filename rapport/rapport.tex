\documentclass{article}
\usepackage{graphicx}

\title{Sciences des données - TP2}
\author{Alexandre Theisse \and Louis-Vincent Capelli \and Tom Sartori}

\begin{document}

\maketitle
\newpage

\section*{Question 1}

\section*{Question 2}

\section*{Question 3}

\subsection*{Mesure de séparation des clusters}
Voici les résultats obtenus pour les 2 mesures de séparation des clusters
pour les différents nombres de clusters.

On constate que le score silhouette est maximal pour 2 clusters et
diminue constamment par la suite. Même le score maximal (0.43) 
est plus proche de 0 que de 1,
ce qui indique que les clusters sont assez mal séparés. Les scores
correspondants aux grands nombres de clusters sont encore plus faibles
(proches de 0.2), ce qui indique un fort chevauchement entre les clusters.

On constate également cette tendance à l'augmentation du chevauchement
en observant les matrices d'overlaps. On remarque que les valeurs
augmentent avec le nombre de clusters, ce qui indique que les clusters
sont de plus en plus chevauchants. En effet, les meilleures valeurs
d'overlaps obtenues entre 2 clusters sont supérieures à 1, ce qui indique
que les clusters ne sont pas séparés et les moins bonnes valeurs
sont de l'ordre de 3 ce qui indique un chevauchement important.

\subsubsection*{2 clusters}

Silhouette score : 0.43 \\\\
Overlaps 2 à 2 :\\

\begin{tabular}{|c|c|c|}
\hline
& C1 & C2 \\
\hline
C1 & - & 4.4 \\
\hline
C2 & - & - \\  
\hline
\end{tabular} 

\subsubsection*{3 clusters}

Silhouette score : 0.34 \\\\
Overlaps 2 à 2 :\\

\begin{tabular}{|c|c|c|c|}
\hline
& C1 & C2 & C3 \\
\hline
C1 & - & 4.49 & 3.38 \\
\hline
C2 & - & - & 1.74 \\
\hline
C3 & - & - & - \\   
\hline
\end{tabular} 

\subsubsection*{4 clusters}

Silhouette score : 0.30 \\\\
Overlaps 2 à 2 :\\

\begin{tabular}{|c|c|c|c|c|}
\hline
& C1 & C2 & C3 & C4 \\
\hline
C1 & - & 1.41 & 4.51 & 3.61 \\
\hline
C2 & - & - & 1.09 & 3.34 \\
\hline
C3 & - & - & - & 1.6 \\
\hline
C4 & - & - & - & - \\
\hline
\end{tabular}

\subsubsection*{5 clusters}

Silhouette score : 0.26 \\\\
Overlaps 2 à 2 :\\

\begin{tabular}{|c|c|c|c|c|c|}
\hline
& C1 & C2 & C3 & C4 & C5 \\
\hline
C1 & - & 3.45 & 1.7 & 1.36 & 4.5 \\
\hline
C2 & - & - & 1.02 & 3.36 & 1.47 \\
\hline
C3 & - & - & - & 0.82 & 4.21 \\
\hline
C4 & - & - & - & - & 0.98 \\
\hline
C5 & - & - & - & - & - \\
\hline
\end{tabular}

\subsubsection*{6 clusters}

Silhouette score : 0.22

\subsubsection*{7 clusters}

Silhouette score : 0.20

\subsubsection*{8 clusters}

Silhouette score : 0.20

\subsubsection*{9 clusters}

Silhouette score : 0.19

\subsubsection*{10 clusters}

Silhouette score : 0.19

\subsection*{Cas particulier des 3 clusters}


\section*{Question 4}

\subsection*{Différentes valeurs de seuil}

En utilisant un clustering hierarchique avec la distance euclidienne et 
le lien de Ward, on obtient peut tester différentes valeurs de seuil pour
obtenir un nombre de clusters donné. Voici les résultats obtenus pour les
différentes valeurs de seuil testées.

NB : Les résultats sont aussi mauvais en utilisant un lien simple ou
moyen.

\subsubsection*{Seuil = 18}
Pour un seuil de 18, on obtient 2 clusters.\\
Voici le dendrogramme obtenu :

\begin{center}
    \includegraphics[scale=0.2]{./img/dendrogram\_with\_threshold\_18\_2\_clusters.png}
\end{center}

\subsubsection*{Seuil = 12}
Pour un seuil de 12, on obtient 3 clusters.\\
Voici le dendrogramme obtenu :

\begin{center}
    \includegraphics[scale=0.2]{./img/dendrogram\_with\_threshold\_12\_3\_clusters.png}
\end{center}

\subsubsection*{Seuil = 9}
Pour un seuil de 9, on obtient 4 clusters.\\
Voici le dendrogramme obtenu :

\begin{center}
    \includegraphics[scale=0.2]{./img/dendrogram\_with\_threshold\_9\_4\_clusters.png}
\end{center}

\subsubsection*{Seuil = 8}
Pour un seuil de 8, on obtient 5 clusters.\\
Voici le dendrogramme obtenu :

\begin{center}
    \includegraphics[scale=0.2]{./img/dendrogram\_with\_threshold\_8\_5\_clusters.png}
\end{center}

\subsubsection*{Seuil = 6.2}
Pour un seuil de 6.2, on obtient 6 clusters.\\
Voici le dendrogramme obtenu :

\begin{center}
    \includegraphics[scale=0.2]{./img/dendrogram\_with\_threshold\_6.2\_6\_clusters.png}
\end{center}

\subsubsection*{Seuil = 6}
Pour un seuil de 6, on obtient 7 clusters.\\
Voici le dendrogramme obtenu :

\begin{center}
    \includegraphics[scale=0.2]{./img/dendrogram\_with\_threshold\_6\_7\_clusters.png}
\end{center}

\subsubsection*{Seuil = 5}
Pour un seuil de 5, on obtient 8 clusters.\\
Voici le dendrogramme obtenu :

\begin{center}
    \includegraphics[scale=0.2]{./img/dendrogram\_with\_threshold\_5\_8\_clusters.png}
\end{center}

\subsubsection*{Seuil = 4.5}
Pour un seuil de 4.5, on obtient 9 clusters.\\
Voici le dendrogramme obtenu :

\begin{center}
    \includegraphics[scale=0.2]{./img/dendrogram\_with\_threshold\_4.5\_9\_clusters.png}
\end{center}

\subsubsection*{Seuil = 3.63}
Pour un seuil de 3.63, on obtient 10 clusters.\\
Voici le dendrogramme obtenu :

\begin{center}
    \includegraphics[scale=0.2]{./img/dendrogram\_with\_threshold\_3.63\_10\_clusters.png}
\end{center}


\subsection*{Mesure de séparation des clusters}
Voici les résultats obtenus pour les 2 mesures de séparation des clusters
pour les différents nombres de clusters.

On constate que le score silhouette est maximal pour 2 clusters et
diminue constamment par la suite. Même le score maximal (0.45) 
est plus proche de 0 que de 1,
ce qui indique que les clusters sont assez mal séparés. Les scores
correspondants aux grands nombres de clusters sont encore plus faibles
(proches de 0.2), ce qui indique un fort chevauchement entre les clusters.

On constate un overlap qui explose entre certains clusters. Par exemple,
dans une configuration avec 2 clusters on obtient un overlap de 3667.46.
\subsubsection*{Seuil = 18}
Pour un seuil de 18, on obtient 2 clusters. Le score silhouette est de 0.45.\\\\
Les overlaps 2 à 2 sont les suivants :\\

\begin{tabular}{|c|c|c|}
\hline
& C1 & C2 \\
\hline
C1 & - & 3667.46 \\
\hline
C2 & - & - \\
\hline
\end{tabular}

\subsubsection*{Seuil = 12}
Pour un seuil de 12, on obtient 3 clusters. Le score silhouette est de 0.28.\\\\
Les overlaps 2 à 2 sont les suivants :\\

\begin{tabular}{|c|c|c|c|}
\hline
& C1 & C2 & C3 \\
\hline
C1 & - & 3667.46 & 7802.60 \\
\hline
C2 & - & - & 1.41 \\
\hline
C3 & - & - & - \\
\hline
\end{tabular}

\subsubsection*{Seuil = 9}

Pour un seuil de 9, on obtient 4 clusters. Le score silhouette est de 0.25.\\\\
Les overlaps 2 à 2 sont les suivants :\\

\begin{tabular}{|c|c|c|c|c|}
\hline
& C1 & C2 & C3 & C4 \\
\hline
C1 & - & 3667.46 & 7802.60 & 954.16 \\
\hline
C2 & - & - & 1.41 & 2.53 \\
\hline
C3 & - & - & - & 4.21 \\
\hline
C4 & - & - & - & - \\
\hline
\end{tabular}

\subsubsection*{Seuil = 8}
Pour un seuil de 8, on obtient 5 clusters. Le score silhouette est de 0.22.\\\\
Les overlaps 2 à 2 sont les suivants :\\

\begin{tabular}{|c|c|c|c|c|c|}
\hline
& C1 & C2 & C3 & C4 & C5 \\
\hline
C1 & - & 1954.64 & 1695.18 & 7802.6 & 954.16 \\
\hline
C2 & - & - & 1.08 & 2.19 & 1.60 \\
\hline
C3 & - & - & - & 2.48 & 1.44 \\
\hline
C4 & - & - & - & - & 4.21 \\
\hline
C5 & - & - & - & - & - \\
\hline
\end{tabular}

\subsubsection*{Seuil = 6.2}
Pour un seuil de 6.2, on obtient 6 clusters. Le score silhouette est de 0.18.

\subsubsection*{Seuil = 6}
Pour un seuil de 6, on obtient 7 clusters. Le score silhouette est de 0.15.

\subsubsection*{Seuil = 5}
Pour un seuil de 5, on obtient 8 clusters. Le score silhouette est de 0.14.

\subsubsection*{Seuil = 4.5}
Pour un seuil de 4.5, on obtient 9 clusters. Le score silhouette est de 0.13.

\subsubsection*{Seuil = 3.63}
Pour un seuil de 3.63, on obtient 10 clusters. Le score silhouette est de 0.13.


\end{document}
